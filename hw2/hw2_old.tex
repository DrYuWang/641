\documentclass[12pt]{amsart}

\usepackage{mathrsfs, fullpage, amsmath, amssymb, graphicx, xcolor, tikz}

\renewcommand{\phi}{\varphi}
\renewcommand{\epsilon}{\varepsilon}
\renewcommand{\hat}{\widehat}
\renewcommand{\tilde}{\widetilde}
\renewcommand{\bar}{\overline}
\newcommand{\bx}{\boldsymbol{x}}
\newcommand{\by}{\boldsymbol{y}}
\newcommand{\cA}{\mathscr{A}}
\newcommand{\cB}{\mathscr{B}}
\newcommand{\cC}{\mathscr{C}}
\newcommand{\cD}{\mathscr{D}}
\newcommand{\cF}{\mathscr{F}}
\newcommand{\cG}{\mathscr{G}}
\newcommand{\cL}{\mathscr{L}}
\newcommand{\cN}{\mathscr{N}}
\newcommand{\cO}{\mathscr{O}}
\newcommand{\cP}{\mathscr{P}}
\newcommand{\cX}{\mathscr{X}}
\newcommand{\cY}{\mathscr{Y}}
\newcommand{\PP}{\mathbb{P}}
\newcommand{\RR}{\mathbb{R}}
\newcommand{\ZZ}{\mathbb{Z}}
\newcommand{\lra}{\longrightarrow}
\newcommand{\One}{\mathbf{1}}
\newcommand{\blank}{\,\cdot\,}

\newcommand{\vpi}{\boldsymbol{\pi}}
\newcommand{\vtheta}{\boldsymbol{\theta}}
\newcommand{\vzero}{\boldsymbol{0}}
\newcommand{\va}{\boldsymbol{a}}
\newcommand{\vb}{\boldsymbol{b}}
\newcommand{\ve}{\boldsymbol{e}}
\newcommand{\vf}{\boldsymbol{f}}
\newcommand{\vg}{\boldsymbol{g}}
\newcommand{\vu}{\boldsymbol{u}}
\newcommand{\vv}{\boldsymbol{v}}
\newcommand{\vw}{\boldsymbol{w}}
\newcommand{\vx}{\boldsymbol{x}}
\newcommand{\vy}{\boldsymbol{y}}
\newcommand{\vz}{\boldsymbol{z}}
\newcommand{\vU}{\boldsymbol{U}}
\newcommand{\vV}{\boldsymbol{V}}
\newcommand{\vW}{\boldsymbol{W}}
\newcommand{\vX}{\boldsymbol{X}}
\newcommand{\vY}{\boldsymbol{Y}}
\newcommand{\vZ}{\boldsymbol{Z}}

\newcommand{\xhat}{\hat{x}}
\newcommand{\yhat}{\hat{y}}
\newcommand{\bxhat}{\hat{\bx}}
\newcommand{\byhat}{\hat{\by}}
\newcommand{\betahat}{\hat{\beta}}

\newcommand{\iid}{i.i.d.\ }

\newcommand{\simiid}{\overset{\text{\textsc{IID}}}\sim}

\DeclareMathOperator{\CV}{CV}
\DeclareMathOperator{\softmax}{softmax}
\DeclareMathOperator{\rank}{rank}
\DeclareMathOperator{\diag}{diag}
\DeclareMathOperator{\expit}{expit}
\DeclareMathOperator{\logit}{logit}
\DeclareMathOperator{\Prob}{Prob}
\DeclareMathOperator{\Bias}{Bias}
\DeclareMathOperator{\EE}{\mathbb{E}}
\DeclareMathOperator{\EPE}{EPE}
\DeclareMathOperator{\Cov}{Cov}
\DeclareMathOperator{\cov}{cov}
\DeclareMathOperator{\Var}{Var}
\DeclareMathOperator{\Mult}{Mult}
\DeclareMathOperator{\Ber}{Ber}
\DeclareMathOperator{\var}{var}
\DeclareMathOperator{\mean}{mean}
\DeclareMathOperator{\MSE}{MSE}
\DeclareMathOperator{\SSE}{SSE}
\DeclareMathOperator{\ESS}{ESS}
\DeclareMathOperator{\RSS}{RSS}
\DeclareMathOperator{\TSS}{TSS}
\DeclareMathOperator*{\argmin}{argmin}
\DeclareMathOperator{\Argmin}{\mathop{\argmin}}


\newtheorem{theorem}{Theorem}
\newtheorem{lemma}[theorem]{Lemma}
\newtheorem{corollary}[theorem]{Corollary}
\newtheorem{thmdef}[theorem]{Theorem-Definition}
\newtheorem{definition}[theorem]{Definition}
\theoremstyle{remark}
\newtheorem{remark}[theorem]{Remark}
\newtheorem{example}[theorem]{Example}


\setlength\parskip{0.5em}
\setlength\parindent{0em}

\definecolor{orange}{HTML}{FF7F0E}
\definecolor{green}{HTML}{2CA02C}
\definecolor{blue}{HTML}{1F77B4}

\DeclareRobustCommand\orangeline{\raisebox{0.5ex}{\tikz \draw[orange, thick] (1, 0) -- (0.5, 0);}}
\DeclareRobustCommand\blueline{\raisebox{0.5ex}{\tikz \draw[blue, thick] (1, 0) -- (0.5, 0);}}

\newcommand{\sol}{\bigskip\noindent\textbf{Solution: }}

\newcommand{\vmu}{\boldsymbol{\mu}}
\newcommand{\vSigma}{\boldsymbol{\Sigma}}
\DeclareMathOperator{\Cat}{Categorical}

\begin{document}
\title{STAT 543/641 -- Winter 2019 -- Homework \#2}

\author{Due March ??, 2019}
\maketitle

Let $\sigma(x)=(1+e^{-x})^{-1}$ be the sigmoid function.
The negative log-likelihood function associated to fitting a univariate logistic regression model to a
dataset $(x_1, y_1),\ldots,(x_n, y_n)$ is
\[
    \ell(a,b) = -\sum_{i=1}^n\Big(y_i\log\sigma(a + bx_i)
    + (1-y_i)\log\big(1-\sigma(a + bx_i)\big)\Big).
\]
In this problem we will identify a condition under which $\ell(a,b)$
does not have a global minimum and a codition under which $\ell(a,b)$ has a unique local minimum.
\begin{enumerate}\setlength\itemsep{1em}
    \item Let $\sigma$ be the sigmoid function.
    Prove: 
    \[
        \sigma'(x) = \sigma(x)\big(1-\sigma(x)\big)\tag{$*$}
    \]
    Conclude that $\sigma'(x)>0$ for all $x$.

    \item Let $x$ and $y$ be constants and let
    \[
        g(a, b) = y\log\sigma(a + bx) + (1-y)\log\big(1-\sigma(a + bx)\big).
    \]
    Prove that
    \[
        \nabla g(a,b) = (y-\sigma(a + bx))\begin{bmatrix}1\\x\end{bmatrix}
            \qquad \text{(Hint: Use $(*)$.)}
    \]
    and that
    \[
        \nabla^2g(a,b) = -\sigma'(a+bx)\begin{bmatrix}
            1 & x\\
            x & x^2
        \end{bmatrix}.
    \]
    Show that $\nabla^2 g(a,b)$ is positive-semidefinite but not positive-definite,
    making $g(a,b)$ convex but not strictly convex.

    \item By (2) and the linearity of the Hessian operator $\nabla^2$,
    \[
        \nabla^2 \ell(a,b) = \sum_{i=1}^n Q_i(a,b),\quad\text{where}\qquad
        Q_i(a,b) = \sigma'(a+bx_i)\begin{bmatrix}
            1 & x_i\\
            x_i & x_i^2
        \end{bmatrix}
    \]

    \item Find a basis of the nullspace $N(Q_i(a,b))$ of $Q_i(a,b)$ whose elements do not depend on $a$ and $b$.
    
    \item Suppose that there are indices $i$ and $j$ such that $x_i\neq x_j$.
    Prove that
    \[
        \bigcap_{i=1}^n N(Q_i(a,b)) = \left\{
            \begin{bmatrix}
                0\\0
            \end{bmatrix}
            \right\}.
    \]
    (Hint: Use (4).)

    \item Suppose that there are indices $i$ and $j$ such that $x_i\neq x_j$.
    Show that $\nabla^2 \ell(a,b)$ is positive-definite and, hence, that $\ell(a,b)$ is strictly convex.
    
    \item Conclude that if that there are indices $i$ and $j$ such that $x_i\neq x_j$, then
    maximum likelihood estimates for $\hat a$ and $\hat b$ are unique if they exist.
\end{enumerate}

\[
    \nabla^2 \ell(a, b) = \sum_{i=1}^n Q_i(a,b),\quad\text{where}\qquad
    Q_i(a,b) = \sigma'(a+bx_i)\begin{bmatrix}
        1 & x_i\\
        x_i & x_i^2
    \end{bmatrix}
\]
where
\[
    Q_i(a,b) = \sigma'(a+bx_i)\begin{bmatrix}
        1 & x_i\\
        x_i & x_i^2
    \end{bmatrix}
\]

\[
    \sigma'(x) = \sigma(x)(1-\sigma(x))
\]

\begin{align*}
    \frac{\partial g}{\partial a} &= y\frac{\sigma'(a + bx)}{\sigma(a + bx)} + (1-y)\frac{(-\sigma'(a + bx))}{1-\sigma(a + bx)}\\
    &= y(1-\sigma(a + bx)) - (1-y)\sigma(a + bx)\\
    &= y - y\sigma(a + bx) - \sigma(a + bx) + y\sigma(a + bx)\\
    &= y-\sigma(a + bx)
\end{align*}


\begin{align*}
    \frac{\partial g}{\partial b} &= y\frac{\sigma'(a + bx)x}{\sigma(a + bx)} + (1-y)\frac{(-\sigma'(a + bx)x)}{1-\sigma(a + bx)}\\
    &= xy(1-\sigma(a + bx)) - x(1-y)\sigma(a + bx)\\
    &= xy - xy\sigma(a + bx) - x\sigma(a + bx) + xy\sigma(a + bx)\\
    &= x(y-\sigma(a + bx))
\end{align*}

\begin{align*}
    g''(b) &= -x^2\sigma'(a + bx)\\
    &= -x^2\sigma(a + bx)(1-\sigma(a + bx))
\end{align*}

\[
    \ell(a,b) = -\log L(a,b) = -\sum_{i=1}^n\Big(y_i\log\sigma(a + bx_i) + (1-y_i)\log\big(1-\sigma(a + bx_i)\big)\Big)
\]
\[
    \ell''(a,b) = \sum_{i=1}^nx_i^2\sigma(a + bx_i)\big(1-\sigma(a + bx_i)\big)
\]
\bigskip
\hrule
\bigskip

For a vector $v\in\RR^2$, write $H_v$ for the open half plane
\[
    H_v = \{w\in\RR^2 : v\cdot w > 0\}.
\]
$H_v$ is the connected component of $\RR^2 - v^\perp$ containing $v$ itself,
$v^\perp$ being the orthogonal complement of $v$:
\[
    v^\perp = \{w\in\RR^2 : v\cdot w = 0\}
\]
Write $C_{v,w}$ for the open cone spanned by vectors $v, w\in\RR^2$, $v\neq w$:
\[
    C_{v,w} = \{av + bw : a,b>0\}
\]
\begin{enumerate}\setlength\itemsep{1em}
    \item Let $u, v, w\in\RR^2$ be three distinct vectors.
    Prove that the following statements are equivalent:
    \begin{enumerate}\setlength\itemsep{0.5em}
        \item $-u\in C_{v,w}$
        \item $-u\in C_{v,w}$, $-w\in C_{u,v}$ and $-v\in C_{w,u}$
        \item $H_u\cap H_v\cap H_w = \varnothing$
        \item $H_u\cup H_v\cup H_w = \RR^2$
    \end{enumerate}

    \medskip\noindent
    If you're having trouble writing up a formal argument here,
    you can draw me a convincing, pretty diagram instead.
    (If you write up a proof, you can optionally include a picture!)

    \item Let $x\in\RR$. Show that the sets
    \[
    \left\{\begin{bmatrix}a\\b\end{bmatrix}: \lim_{t\to\infty}\sigma(ta + tbx)=0\right\}
        \quad\text{and}\quad
    \left\{\begin{bmatrix}a\\b\end{bmatrix}: \lim_{t\to\infty}\big(1 - \sigma(ta + tbx)\big)=0\right\}
    \]
    have the form $H_{\pm v}$, where $v=\begin{bmatrix}
        1\\x
    \end{bmatrix}$.

    \item For $K>0$, identify the level curves
    \[
    \left\{\begin{bmatrix}a\\b\end{bmatrix}: \sigma(a + bx)=K\right\}
        \quad\text{and}\quad
    \left\{\begin{bmatrix}a\\b\end{bmatrix}: \big(1 - \sigma(a + bx)\big)=K\right\}
    \]
    as lines
    and the level sets
    \[
    \left\{\begin{bmatrix}a\\b\end{bmatrix}: \sigma(a + bx)\leq K\right\}
        \quad\text{and}\quad
    \left\{\begin{bmatrix}a\\b\end{bmatrix}: \big(1 - \sigma(a + bx)\big)=K\right\}
    \]
    as half-plane translates of the form $y+H_v$ and $z+H_w$.

    \item Consider a three point data set $\{(x_1, y_1), (x_2, y_2), (x_3, y_3)\}$,
    where $x_1<x_2<x_3$ and let
    \[
        L(a,b)=\prod_{i=1}^3 p(x_i, y_i|a, b),
    \]
    where
    \[
        p(x_i, y_i|a, b) = \sigma(a + bx_i)^{y_i}(1 - \sigma(a + bx_i)^{y_i})^{1-y_i}.
    \]
    Show that
    \[
        \lim_{a^2+b^2\to\infty} L(a,b) = 0
    \]
    if and only if $y_1=y_3=1$ and $y_2=0$ or $y_1=y_3=0$ and $y_2=0$.
    For $0<K<1$, consider the sets
    \[
        S_{i,K} = \{(a, b) : p_i(x_i, y_i|a,b) > K\},\qquad i=1,2,3.
    \]
    Prove that $S_{1,K}\cap S_{2,K}\cap S_{3,K}$ is bounded if and only if
    either $y_1 = y_3 = 0$ and $y_2=1$ or $y_1=y_3=1$ and $y_2=0$.
    
    
    Let
    \[
        v_i = \begin{bmatrix}
            1\\x_i
        \end{bmatrix},\qquad i=1,2,3.
    \]
    Prove that
    \[
        L(a,b)
    \]
    \begin{enumerate}
    \item Suppose $y_1=y_3=0$ and $y_2=1$. 
    \end{enumerate}
    
\end{enumerate}

\bigskip
\hrule
\bigskip


Consider fitting a dataset $(x_1, y_1),\ldots,(x_n, y_n)$ to
the following simplified logistic regression model:
\[
    Y_i|X=x_i \sim p(x_i, y_i) := \sigma(bx_i)^{y_i}(1 - \sigma(bx_i))^{1-y_i}
\]
Here, $\sigma$ denotes the sigmoid function: $\sigma(t) = (1 + e^{-t})^{-1}$.
Let $L(b)$ the the likelihood function associated this dataset/model.

Suppose that our dataset has the following property:
\[
    \text{$x_i<0$ if and only if $y_i=0$}\tag{$*$}
\]
Prove that
\[
    \lim_{b\to\infty}L(b) = 1.
\]
What can we say about $\argmax_b L(b)$ in this case?

\sol

Suppose that $x_i<0$ and $y_i=0$. Then $e^{-bx_i}\to \infty$ as $b\to\infty$. It follows that
\begin{align*}
    \lim_{b\to\infty}p(x_i, y_i) =  \lim_{b\to\infty} \big(1 - \sigma(bx_i)) = 1 - 0 = 1.
\end{align*}
If $x_i>0$ and $y_i=1$, then $e^{-bx_i}\to 0$ as $b\to\infty$. It follows that
\[
    \lim_{b\to\infty}p(x_i, y_i) =  \lim_{b\to\infty} \sigma(bx_i) = 1.
\]
Therefore, by property $(*)$,
\[
    \lim_{b\to\infty} p(x_i, y_i)=1
\]
for all $i=1,\ldots,n$. Consequently,
\[
    \lim_{b\to\infty} L(b) = \lim_{b\to\infty}\prod_{i=1}^n p(x_i, y_i) = 1.
\]


Suppose, now, that there are indices $i$ and $j$ such that
\[
    \text{$x_i<0$ and $y_i=1$}\quad\text{and}\quad
    \text{$x_j>0$ and $y_j=0$}.
\]
In particular, property $(*)$ is not satisfied.
Prove that
\[
    \lim_{b\to\pm\infty}L(b) = 0.
\]
What can we say about $\argmax_b L(b)$ in this case?

\sol
As $y_i=0$, $p(x_i, y_i|b) = \sigma(bx_i)$. As $x_i<0$, $\sigma(bx_i)\to 0$ as $b\to\infty$.
Thus
\[
    \lim_{b\to\infty}p(x_i, y_i|b) =  \lim_{b\to\infty} \sigma(bx_i) = 0.
\]
As $y_j=1$, $p(x_j, y_j|b) = 1-\sigma(bx_j)$. As $x_j>0$, $1-\sigma(bx_j)\to 0$ as $b\to\infty$.
Thus
\[
    \lim_{b\to\infty}p(x_j, y_j|b) =  \lim_{b\to\infty} \sigma(bx_j) = 0.
\]


\bigskip\hrule\bigskip

Let $X_1,\ldots,X_m$ and $Y_1,\ldots,Y_n$ be random samples from normally distributed populations with 
means $\mu_X$ and $\mu_Y$ and variances $\sigma_X^2$ and $\sigma_Y^2$, respectively.
Let $S_X^2$ and $S_Y^2$ be the standard unbiased estimators of $\sigma_X^2$ and $\sigma_Y^2$,
respectively.
\begin{enumerate}\setlength\itemsep{1em}
\item Suppose $\sigma_X^2 = \sigma_Y^2$ and write $\sigma^2$ for this common value.
\[
    S^2 := \frac{(m-1)S_X^2 + (n-1)S_Y^2}{m + n - 2}
\]
is an unbiased estimator of $\sigma^2$. It's called the \emph{pooled variance estimator}.

\sol
Since $S_X^2$ and $S_Y^2$ are unbiased estimators of $\sigma_X$ and $\sigma_Y$, respectively,
\begin{align*}
    \EE[S^2] &= \frac{m-1}{m + n - 2}\EE[S_X^2] + \frac{n-1}{m+n-2}\EE[S_Y^2]\\
    &= \frac{m-1}{m + n - 2}\sigma_X^2 + \frac{n-1}{m+n-2}\sigma_Y^2\\
    &= \frac{m-1}{m + n - 2}\sigma^2 + \frac{n-1}{m+n-2}\sigma^2&\text{(as $\sigma_X^2 = \sigma^2 = \sigma_Y^2$)}\\
    &= \sigma^2
\end{align*}


\item Suppose, in addition to having common variance,
that the $X_i$ are independent of the $Y_i$.
What is the distribution of
\[
\frac{(m+n-2)S^2}{\sigma^2}?
\]
What is the variance of $S^2$?

\sol
Since $S_X^2$ and $S_Y^2$ are independent,
\[
    \frac{(m-1)S_X^2}{\sigma^2}\sim \chi^2_{m-1}
    \quad\text{and}\quad
    \frac{(n-1)S_Y^2}{\sigma^2}\sim \chi^2_{n-1},
\]
and it follows from general properties of $\chi^2$-distributions that
\[
    \frac{(m+n-2)S^2}{\sigma^2}
    = \frac1{\sigma^2}\big((m-1)S_X^2 + (n-1)S_Y^2\big)\sim \chi^2_{m+n-2}.
\]
Since $ \chi^2_{m+n-2}$-distributed random variable has variance $2(m+n-2)$, it follows that
\[
    \Var S^2 = \frac{2\sigma^4}{m+n-2}.
\]


\item[($3^*$)] Generalize these results from the case of $K=2$ populations to that of an arbitrary $K$.
Compare with equation~(4.15) in~\cite{ISLR}.

\item[($4^*$)] Can you prove analogous results with covariance matrices in place of scalar variances?
\end{enumerate}




Since $S_X^2$ and $S_Y^2$ are independent,
\[
    \frac{(m-1)S_X^2}{\sigma^2}\sim \chi^2_{m-1}
    \quad\text{and}\quad
    \frac{(n-1)S_Y^2}{\sigma^2}\sim \chi^2_{n-1},
\]
and it follows from general properties of $\chi^2$-distributions that
\[
    \frac{(m+n-2)S^2}{\sigma^2}
    = \frac1{\sigma^2}\big((m-1)S_X^2 + (n-1)S_Y^2\big)\sim \chi^2_{m+n-2}.
\]
Since $ \chi^2_{m+n-2}$-distributed random variable has variance $2(m+n-2)$, it follows that
\[
    \Var S^2 = \frac{2\sigma^4}{m+n-2}.
\]






    \bigskip\hrule\bigskip
    \cite[Exercise 12.16]{CB}
    This exercise examines an extreme case in which the likelihood equations for logistic regression have no solution.
    
    \noindent
    Consider the following $20$-point data set:
    \begin{multline*}
        (0, 1), (0, 1), (0, 1), (0, 1), (0, 1), (0, 1), (0, 1), (0, 1), (0, 1), (0, 1)\\
        (1, 0), (1, 0), (1, 0), (1, 0), (1, 0), (1, 1), (1, 1), (1, 1), (1, 1), (1, 1)
    \end{multline*}
    \begin{enumerate}\setlength\itemsep{0.5em}
        \item Observe that, empirically, $\Prob(Y=1|X=0)=1$ and $\Prob(Y=1|X=1)=0.5$.
    Let $\sigma(t)=(1+e^{-t})^{-1}$ be the sigmoid function.
    Are there $a$ and $b$ such that
    $\sigma(a + b\cdot 0) = 1$ or $\sigma(a+b\cdot 1)=0.5$?

    \sol
    \[
        \sigma(x) = \frac{1}{1 + e^{-x}}
    \]
    Since the exponential function takes values in $(0,\infty)$,
    \[
        0< \sigma(x) < 1,
    \]
    for every $x\in\RR$. In particular, there are no numbers $a$ and $b$ such that
    $\sigma(a + b\cdot 0) = 1$.
    Clearly, $\sigma(0) = \sigma(0+0\cdot 1) = 0.5$.

    \item Let $\cL(a,b)$ be the likelihood function associated to fitting a
    logistic regression model to this data set. Show that
    \[
        L := \lim_{b\to\infty}\cL(-b, b) = \sup_{(a, b)\in\RR^2}\cL(a,b) < \infty
    \]
    and that $\cL(a,b)\neq L$ for any $(a,b)\in\RR^2$.
    What are
    \[
        \lim_{b\to\infty}\sigma(-b + b\cdot 0)\quad\text{and}\quad
        \lim_{b\to\infty}\sigma(-b + b\cdot 1)?
    \]
    \end{enumerate}


    Let $(\vX, Y)$ be jointly distributed,
    where $\vX$ is a $p$-dimensional random vector and $Y$ takes values in $\{1,\ldots,K\}$.
    Suppose that, for each $k$, $\vX|Y=k$ has Gaussian distribution with mean $\vmu_k$ and
    and variance $\Sigma$, with the latter independent of $k$.

    Consider a data set $(\vx^{(1)}, y^{(1)}),\ldots,(\vx^{(n)}, y^{(n)})$, where
    $\vx^{(i)}\in\RR^{1\times p}$ and $y^{(i)}\in\{1,\ldots,K\}$.
    For $1\leq k\leq K$, let
    \[
        I_k = \{i : y^{(i)} = k\},\quad n_k = |I_k|,\quad \hat\pi_k = \frac{n_k}{n}.
    \]
    Define sample means $\vmu_k$ and a pooled sample covariance $\vSigma$ by
    \begin{align*}
        \hat \vmu_k =\hat\vmu_{k,\vx}&= \frac1{n_k}\sum_{i\in I_k} \vx^{(i)}\in\RR^{p\times 1},\\
        \hat \vSigma =\hat\vSigma_{\vx}&= \frac1{n-K}\sum_{k=1}^K\sum_{i\in I_k}
        (\vx^{(i)} - \hat \vmu_k)^T(\vx^{(i)} - \hat \vmu_k)\in\RR^{p\times p}.
    \end{align*}
    Define linear discriminant functions, $\delta_k=\delta_{k,\vx}$, by
    \[
        \delta_k(\vv)=\delta_{k,\vx}(\vv) = \vv\hat\vSigma\hat\vmu_k^T - \frac12 \hat\vmu_k\hat\vSigma\hat\vmu_k^T + \log\hat\pi_k,
        \quad \vv\in\RR^{p\times 1}.
    \]
    
    Let $\va\in\RR^{p\times 1}$ and let
    \[
        \vw^{(i)} = \vx^{(i)} - \va.
    \]
    \[
        \hat\vmu_{k,\vw} = \hat\vmu_{k,\vx} - \va,\quad
        \Sigma_w = \Sigma_x
    \]
    \[
        \delta_{k_1,w}(v - a) - \delta_{k_2, w}(v-a) = \delta_{k_1,x}(v) - \delta_{k_2,x}(v) 
    \]
    Let $U\in\RR^{p\times p}$ be an orthogonal matrix and let $w^{(i)}=Ux^{(i)}$.
    Then
    \[
        \delta_{k, Ux}(Uv) = \delta_x(v).
    \]
    \[
        \sum_{k=1}^K\pi_k\mu_k = \mu
    \]
    
    \bigskip
    \hfill
    \bigskip

    Suppose
    \begin{align*}
        Y&\sim \Cat\left(\tfrac1K,\ldots,\tfrac1K\right),\\
        \vX|Y=k&\sim N(\vmu_k,\sigma I).
    \end{align*}
    Show that
    \[
        \argmin_k p(Y=k|\vX=\vx) = \argmin_k \|\vx - \vmu_k\|.
    \]

    \bigskip
    \hfill
    \bigskip

    Logistic regression (with and without ridge regularization, with and without PCA), LDA, Gaussian na\"ive Bayes, for breast cancer data set. Plot in 2d with decision boundary. Optional: Lasso

    Document classification with multinomial na\"ive Bayes

    Ridge regression via constrained optimization.


    \section{Total variation $=$ variance within $+$ variance between}

    Let $X$ and $Y$ be jointly distributed random variables, where
    \begin{align*}
        Y &\sim \Cat(\pi_1,\ldots,\pi_K), & &\sum \pi_k=1,\\
        X\mid Y=k &\sim N(\mu_k, \sigma_k^2), & &k=1,\ldots,K.
    \end{align*}
    Assume that the $\mu_k$ are pairwise distinct.
    Prove that
    \[
        \EE[X] = \sum_{k=1}^K \pi_k\mu_k
    \]
    and that
    \[
        \Var X = \sum_{k=1}^K \pi_k\sigma_k^2 + \sum_{k=1}^K \pi_k(\mu_k - \mu)^2.
    \]
    To establish the decomposition of the variance, you might want to use the
    \emph{law of total variation:}
    \[
    \Var X = \EE(\Var(X|Y)) + \Var \EE(X|Y).
    \]

    The random variable $X$ has marginal density $p(x)$, where
    \[
        p(x) = \sum_{k=1}^K p(x,k) = \sum_{k=1}^K p(k)p(x|k) = \sum_{k=1}^K \pi_k p(x|k).
    \]
    Therefore,
    \[
        \EE[X] = \sum_{k=1}^K \pi_k \EE[X_k]
    \]
    where $X_k$ is a random variable with density $p(x|k)$.
    By hypothesis, $X_k$ has expected value $\mu_k$. Therefore,
    \[
        \EE[X] = \sum_{k=1}^K \pi_k \mu_k.
    \]

    Since $\Var(X|Y=k) = \sigma_k^2$,
    \[
        \EE[\Var(X|Y)] = \sum_{k=1}^K p(Y=k)\Var(X|Y=k) = \sum_{k=1}^K \pi_k\sigma_k^2.
    \]
    
    Let $\mu=\EE[X]$.
    By the \emph{law of total expectation},
    \[
        \EE[\EE[X|Y]] = \EE[X] = \mu.
    \]
    Let $Z=\EE[X|Y]$. Then set of values of $Z$ is $\{\mu_1,\ldots,\mu_K\}$.
    Since the $\mu_k$ are pairwise distinct,
    \[
        \Prob[Z = \mu_k] = p(Y=k) = \pi_k.
    \]
    Therefore,
    \[
        \Var Z = \sum_{k=1}^K \Prob[Z = \mu_k](\mu_k - \mu)^2
        = \sum_{k=1}^K \pi_k\big(\mu_k - \mu\big)^2.
    \]

    Let $x_1,\ldots,x_n$ be numbers and set 
    \[
        \mu = \frac1{n}\sum_{i=1}^n x_i,\qquad 
        \sigma^2 = \frac1{n}\sum_{i=1}^n (x_i - \mu)^2.
    \]
    Suppose that
    \[
        \{1,\ldots,n\} = \bigsqcup_{k=1}^K I_k\qquad \text{(disjoint union)}
    \]
    with
    \[
        n_k := |I_k| > 0.
    \]
    Set
    \[
        \mu_k = \frac1{n_k}\sum_{i\in I_k}^n x_i,\qquad 
        \sigma_k^2 = \frac1{n_k}\sum_{i\in I_k}^n (x_i - \mu_k)^2
    \]
    Prove that
    \[
        \sigma^2 = \sum_{k=1}^K \pi_k\sigma_k^2 + \sum_{k=1}^K \pi_k (\mu_k - \mu)^2,
        \quad\text{where}\quad
        \pi_k = \frac{n_k}n.
    \]
    (This is an ``algebraic version'' of the law of total variance)

    \begin{align*}
        \frac1{n}\sum_{i=1}^n (x_i - \mu)^2 &= \frac1{n}\sum_{k=1}^K \sum_{i\in I_k}(x_i - \mu)^2\\
        &= \frac1{n}\sum_{k=1}^K \sum_{i\in I_k}(x_i - \mu_k + \mu_k - \mu)^2\\
        &= \frac1{n}\sum_{k=1}^K \sum_{i\in I_k}
        \big\{(x_i - \mu_k)^2 + (\mu_k - \mu)^2 + 2(x_i - \mu_k)(\mu_k - \mu)\big\}\\
        &= \frac1{n}\sum_{k=1}^K n_k\frac1{n_k}\sum_{i\in I_k}(x_i - \mu_k)^2
        + \frac1{n}\sum_{k=1}^K(\mu_k - \mu)^2\sum_{i\in I_k}1\\
        &\qquad\qquad + \frac2{n}\sum_{k=1}^K (\mu_k - \mu)\sum_{i\in I_k}(x_i - \mu_k)\\
        &= \sum_{k=1}^K \frac{n_k}n \sigma_k^2 + \sum_{k=1}^K \frac{n_k}n (\mu_k - \mu)^2 + 0\\
        &= \sum_{k=1}^K \pi_k \sigma_k^2 + \sum_{k=1}^K \pi_k (\mu_k - \mu)^2
    \end{align*}

    Define a probability space $(\Omega, \mu)$ by
    \[
        \Omega = \{x_1,\ldots,x_n\}\times \{1,\ldots,K\},\quad
        \mu\big(\{(x_i, k)\}\big)= p_{ik} := \begin{cases}
            \dfrac1n&\text{if $i\in I_k$,}\\[2ex]
            0&\text{otherwise.}
        \end{cases}
    \]
    Note that $\mu$ is, indeed, a probability measure on $\Omega$:
    \[
        \sum_{(i, k)\in\Omega} p_{ik} =\sum_{k=1}^K\sum_{i=1}^n p_{ik}
        = \sum_{k=1}^K\sum_{i\in I_k}\frac1n = \sum_{k=1}^n\frac{n_k}n = 1
    \]
    Define random variables $X$ and $Y$ on $\Omega$ by
    \[
        X(x_i, k) = x_i\quad\text{and}\quad Y(x_i, k) = k.
    \]
    Then
    \[
        \mu = \EE[X] = \sum_{k=1}^K\sum_{i=1}^n p_{ik}x_i
        = \sum_{k=1}^K\sum_{i\in I_k}\frac1n x_i
        = \frac1n\sum_{k=1}^K\sum_{i\in I_k} x_i = \frac1n\sum_{i=1}^n x_i.
    \]
    Let $Z=\EE[X|Y]$, so that
    \[
        Z(k) = \EE[X|Y = k]
        = \frac{\sum_{i=1}^np_{ik}x_i}{\sum_{i=1}^np_{ik}}
        = \frac{\sum_{i\in I_k}x_i}{\sum_{i\in I_k}^n1}
        = \frac1{n_k}\sum_{i\in I_k}x_i = \mu_k.
    \]
    \begin{align*}
        \Var[X|Y=k] &= \EE[(X - Z)^2|Y=k]\\
        &= \frac{\sum_{i=1}^np_{ik}(X(i) - Z(k))^2}{\sum_{i=1}^n p_{ik}}\\
        &= \frac{\frac1n\sum_{i\in I_k}(x_i - \mu_k)^2}{\sum_{i\in I_k} \frac1n}\\
        &= \frac1{n_k}\sum_{i\in I_k}(x_i - \mu_k)^2\\
        &= \sigma_k^2
    \end{align*}

    \begin{align*}
        \EE\left[\Var[X|Y]\right] &= \sum_{k=1}^K\Prob(Y=k)\Var[X|Y=k]\\
        &=\sum_{k=1}^K\left(\sum_{(i,k)}p_{ik}\right)\sigma_k^2\\
        &=\sum_{k=1}^K\frac{n_k}n\sigma_k^2\\
        &=\sum_{k=1}^K\pi_k\sigma_k^2
    \end{align*}
    
    Finally,
    \begin{align*}
        \Var Z &= \EE\left[(Z - \EE[Z])^2\right]\\
        &= \EE\left[(Z - \mu)^2\right] &\text{(by the law of total expectation)}\\
        &= \sum_{k=1}^K\Prob(Z=\mu_k)(\mu_k - \mu)^2\\
        &= \sum_{k=1}^K\Prob(Y=k)(\mu_k - \mu)^2&\text{(as $k= \ell \Longleftrightarrow \mu_k= \mu_\ell$)} \\
        &= \sum_{k=1}^K\left(\sum_{(i, k)}p_{ik}\right)(\mu_k - \mu)^2\\
        &= \sum_{k=1}^K\pi_k(\mu_k - \mu)^2.
    \end{align*}


\subsection{Matrix version}
\begin{align*}
    \frac1{n}\sum_{i=1}^n (x_i - \mu)(x_i - \mu)^T &=
    \frac1{n}\sum_{k=1}^K \sum_{i\in I_k}(x_i - \mu)(x_i - \mu)^T\\
    &= \frac1{n}\sum_{k=1}^K \sum_{i\in I_k}((x_i - \mu_k) + (\mu_k - \mu))((x_i - \mu_k) + (\mu_k - \mu))^T\\
    &= \frac1{n}\sum_{k=1}^K \sum_{i\in I_k}
    \big\{(x_i - \mu_k)(x_i - \mu_k)^T + (\mu_k - \mu)(\mu_k - \mu)^T\\
    &\qquad\qquad+ (x_i - \mu_k)(\mu_k - \mu)^T + (\mu_k - \mu)(x_i - \mu_k)^T\big\}\\
    &= \frac1{n}\sum_{k=1}^K n_k\frac1{n_k}\sum_{i\in I_k}(x_i - \mu_k)(x_i - \mu_k)^T
    + \frac1{n}\sum_{k=1}^K(\mu_k - \mu)(\mu_k - \mu)^T\sum_{i\in I_k}1\\
    &\qquad\qquad + \frac1{n}\sum_{k=1}^K \left\{\sum_{i\in I_k}(x_i - \mu_k)\right\}(\mu_k - \mu)^T\\
    &\qquad\qquad+ \frac1{n}\sum_{k=1}^K (\mu_k - \mu)\sum_{i\in I_k}(x_i - \mu_k)^T\\
    &= \sum_{k=1}^K \frac{n_k}n \Sigma_k^2 + \sum_{k=1}^K \frac{n_k}n (\mu_k - \mu)(\mu_k - \mu)^T + 0 + 0\\
    &= \sum_{k=1}^K \pi_k \Sigma_k^2 + \sum_{k=1}^K \pi_k (\mu_k - \mu)(\mu_k - \mu)^T
\end{align*}




    \begin{thebibliography}{99}
        \bibitem{CB}Casella, Berger, \emph{Statistical Inference (2nd ed.)}, Duxbury, 2002.
        % \bibitem{HMC}Hogg, McKean, Craig, \emph{Introduction to Mathematical Statistics (7th ed.)}, Pearson, 2013.
        \bibitem{ISLR}
    \end{thebibliography}


\end{document}